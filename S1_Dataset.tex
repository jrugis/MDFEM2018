\documentclass[10pt,letterpaper]{article}

\usepackage[margin=1.2in]{geometry}
\usepackage{amsmath,amssymb}

\title{Multi-Domain Finite Element Meshing\\for Parotid Acinar Cell Modeling and Simulation}
\date{}

\begin{document}
\maketitle
\thispagestyle{empty}

\section*{Iteration evolution result data}

\flushleft
\begin{table}[h!]
%\begin{center}
\footnotesize
\begin{tabular}{|c|ccccccccccc|}
\hline
cell & 0 &1 &2 &3 &4 &5 &6 &7 &8 &9 &10\\
\hline

1 &0.9837 &0.7466 &0.5595 &0.4472 &0.3726 &0.3201 &0.2814 &0.2520 &0.2289 &0.2105 &0.1955\\
2 &1.0143 &0.7216 &0.5146 &0.4141 &0.3559 &0.3177 &0.2906 &0.2702 &0.2543 &0.2414 &0.2309\\
3 &1.0620 &0.7552 &0.5703 &0.4641 &0.3963 &0.3501 &0.3169 &0.2921 &0.2727 &0.2571 &0.2443\\
4 &1.0402 &0.7387 &0.5579 &0.4552 &0.3903 &0.3459 &0.3135 &0.2888 &0.2691 &0.2529 &0.2394\\
5 &1.1262 &0.8054 &0.5683 &0.4418 &0.3649 &0.3139 &0.2782 &0.2524 &0.2331 &0.2184 &0.2070\\
6 &1.0739 &0.7598 &0.5541 &0.4453 &0.3766 &0.3286 &0.2927 &0.2649 &0.2428 &0.2249 &0.2102\\
7 &1.0451 &0.7137 &0.5396 &0.4363 &0.3692 &0.3226 &0.2885 &0.2626 &0.2422 &0.2259 &0.2126\\
\hline
\end{tabular}
%\end{center}
\caption{The evolution of standard deviation of mean surface curvature for the seven cells (in $\mu \text{m}^{-1}$)  after each of ten coupled smoothing iterations.}
%\label{tab:curv}
\end{table}

\begin{table}[h!]
%\begin{center}
\footnotesize
\begin{tabular}{|c|ccccccccccc|}
\hline
cell & 0 &1 &2 &3 &4 &5 &6 &7 &8 &9 &10\\
\hline
1 &628.64 &590.17 &569.09 &559.23 &553.53 &549.84 &547.26 &545.37 &543.93 &542.80 &541.90\\
2 &729.06 &680.85 &657.39 &646.44 &639.76 &635.12 &631.65 &628.91 &626.66 &624.77 &623.15\\
3 &681.01 &638.45 &619.64 &609.78 &603.46 &598.93 &595.46 &592.68 &590.37 &588.40 &586.70\\
4 &712.88 &661.36 &637.82 &626.19 &619.08 &614.15 &610.47 &607.57 &605.02 &603.22 &601.52\\
5 &443.18 &408.76 &393.58 &386.72 &382.76 &380.16 &378.30 &376.90 &375.81 &374.92 &374.19\\
6 &684.72 &636.81 &615.75 &605.50 &599.26 &595.00 &591.87 &589.46 &587.55 &585.98 &584.67\\
7 &627.42 &582.15 &564.89 &555.74 &549.94 &545.88 &542.82 &540.42 &538.47 &536.84 &535.46\\
\hline
\end{tabular}
%\end{center}
\caption{The evolution of surface area for the seven cells (in $\mu \text{m}^2$)  at each coupled smoothing iteration.}
%\label{tab:surf}
\end{table}

\begin{table}[h!]
%\begin{center}
\footnotesize
\begin{tabular}{|c|ccccccccccc|}
\hline
cell & 0 &1 &2 &3 &4 &5 &6 &7 &8 &9 &10\\
\hline
1 &1004.3 &1004.0 &1003.9 &1003.9 &1003.8 &1003.7 &1003.7 &1003.7 &1003.7 &1003.7 &1003.7\\
2 &1012.4 &1011.9 &1011.8 &1011.8 &1011.8 &1011.9 &1011.9 &1012.0 &1012.0 &1012.1 &1012.2\\
3 &1090.2 &1089.6 &1089.4 &1089.3 &1089.3 &1089.3 &1089.4 &1089.4 &1089.5 &1089.5 &1089.6\\
4 &1105.7 &1105.4 &1105.3 &1105.3 &1105.4 &1105.4 &1105.5 &1105.5 &1105.6 &1105.6 &1105.6\\
5 &492.76 &492.75 &492.88 &492.98 &493.05 &493.11 &493.16 &493.20 &493.23 &493.26 &493.29\\
6 &1036.3 &1036.2 &1036.2 &1036.1 &1036.1 &1036.1 &1036.1 &1036.1 &1036.1 &1036.2 &1036.2\\
7 &903.99 &903.60 &903.50 &903.53 &903.60 &903.69 &903.78 &903.87 &903.94 &904.01 &904.07\\
\hline
\end{tabular}
%\end{center}
\caption{The evolution of volume for the seven cells (in $\mu \text{m}^3$)  at each coupled smoothing iteration.}
%\label{tab:vol}
\end{table}

\end{document}

